\documentclass[12pt]{article}

\usepackage{sbc-template}
\usepackage{graphicx,url}
\usepackage[utf8]{inputenc}
\usepackage[brazil]{babel}
\usepackage{lipsum}
%\usepackage[latin1]{inputenc}

\sloppy

\title{Aplicativo de coleta de inventário\\ Título provisório}

\author{Ivaney Vieira de Sales\inst{1}, Ricardo Ramos\inst{2}}

\address{Instituto Federal do Piauí (IFPI)\\
  R. Álvaro Mendes, 94 -- Centro (Sul), Teresina -- PI, 64000-040
\nextinstitute
Instituto Federal do Piauí (IFPI)\\
R. Álvaro Mendes, 94 -- Centro (Sul), Teresina -- PI, 64000-040
}

\begin{document} 

\maketitle

\begin{resumo}

  /* Será a última coisa a ser feita */
  
\end{resumo}

\begin{abstract}

  /* It will be the last thing to be done */

\end{abstract}

\section{Introdução}

% apresentar o problema que o aplicativo visa solucionar

No cenário dinâmico do varejo brasileiro, os erros de estoque representam uma questão crucial que pode impactar negativamente a eficiência operacional e a satisfação do cliente. O desafio reside na necessidade de manter um equilíbrio delicado entre a oferta e a demanda, garantindo que os produtos certos estejam disponíveis no momento certo. Nesse contexto, a implementação de um processo de inventário assume um papel estratégico e indispensável.

% Um breve referencial teórico

Realizar inventários é crucial para garantir a precisão das informações de saldo de estoque. Erros no registro de transações e no manuseio físico do estoque podem causar discrepâncias entre o estoque registrado e o real, que só são corrigidas durante verificações físicas esporádicas. Na prática, diversas transações aumentam a possibilidade de erros, resultando em registros imprecisos de estoque. As causas comuns incluem erros de digitação, contagem incorreta de produtos, falhas em registrar corretamente produtos danificados ou destruídos, retirada e retorno de itens sem a devida correção nos registros, atrasos na atualização dos registros após transações e roubos de estoque, que são frequentes no varejo e também presentes em ambientes industriais e comerciais \cite{nigel}.

Segundo a pesquisa realizada por \cite{silva}, a adoção de inventários cíclicos representa uma prática estratégica que não apenas melhora a acurácia e a gestão dos estoques, mas também promove a eficiência operacional e a redução de custos, fortalecendo a competitividade da empresa no mercado.

% Objetivo final da introdução


\section{Metodologia}

Descrever detalhadamente a metodologia utilizada no desenvolvimento do aplicativo,

incluindo:

* Abordagem de desenvolvimento: (Ex: Ágil, Waterfall)

* Ferramentas e tecnologias: (Ex: Linguagens de programação, frameworks, ferramentas de design)

* Coleta de dados (caso tenha sido utilizada): (Ex: Entrevistas, questionários, testes de usabilidade)

* Análise de dados (se for utilizada no trabalho): (Ex: Métodos estatísticos, análise de conteúdo)


\section{Resultado e Discussão}

% Descrever as funcionalidades do aplicativo e as telas (figuras) criadas e como elas atendem aos objetivos da pesquisa

% Apresentar trabalhos relacionados (similares) ao aplicativo desenvolvido e informar o diferencial do aplicativo desenvolvido


\section{Considerações finais}

/****************************************/

/*Rascunho*/

/****************************************/

O inventário não deve ser encarado apenas como uma correção pontual, mas como um processo contínuo e integrado à gestão operacional. A implementação de tecnologias avançadas, como sistemas automatizados de rastreamento e leitura por código de barras, pode elevar ainda mais a eficácia desse processo, reduzindo o tempo necessário para identificar e corrigir discrepâncias.

Em resumo, o problema dos erros de estoque no varejo brasileiro é uma realidade desafiadora, mas o investimento em um processo de inventário robusto emerge como uma solução estratégica e proativa. Ao adotar essa abordagem, os varejistas não apenas minimizam as falhas operacionais, mas também fortalecem a base para uma gestão de estoque eficiente e uma experiência de compra mais satisfatória para os clientes.


\section{Referencia}



Bibliographic references must be unambiguous and uniform.  We recommend giving
the author names references in brackets, e.g. \cite{knuth:84},
\cite{boulic:91}, and \cite{smith:99}.

Eliane Francisca Silva. Método DMAIC aplicado ao controle de estoques de telefonia e informática numa empresa de varejo [[recurso eletrônico]]. Piracicaba, SP: ESALQ/USP; 2021.

Slack, N.; Chambers, S.; Johnston, R. Administração da produção. 8ed. São Paulo: Atlas, 2018.

\bibliographystyle{sbc}
\bibliography{TCC_Ivaney_Vieira_de_Sales}

\end{document}
