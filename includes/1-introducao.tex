% apresentar o problema que o aplicativo visa solucionar

No varejo brasileiro, erros de estoque impactam negativamente a eficácia dos processos e a satisfação do cliente.
O desafio reside na necessidade de manter um equilíbrio delicado entre a oferta e a demanda, garantindo que os produtos certos estejam disponíveis no momento certo. Nesse contexto, a implementação de um processo de inventário assume um papel estratégico e indispensável.

% Um breve referencial teórico

Realizar inventários é crucial para garantir a precisão das informações de saldo de estoque. Erros no registro de transações e no manuseio físico do estoque podem causar discrepâncias entre o estoque registrado e o real, que só são corrigidas durante verificações físicas esporádicas. Na prática, diversas transações aumentam a possibilidade de erros, resultando em registros imprecisos de estoque. As causas comuns incluem erros de digitação, contagem incorreta de produtos, falhas em registrar corretamente produtos danificados ou destruídos, retirada e retorno de itens sem a devida correção nos registros, atrasos na atualização dos registros após transações e roubos de estoque, frequentes no varejo e também presentes em ambientes industriais e comerciais \cite{nigel}.

Segundo a pesquisa realizada por \cite{silva}, a adoção de inventários cíclicos representa uma prática estratégica que não apenas melhora a acurácia e a gestão dos estoques, mas também promove a eficiência operacional e a redução de custos, fortalecendo a competitividade da empresa no mercado.

% Objetivo geral da introdução

Um aplicativo de gestão de inventário emerge como uma solução indispensável para o controle e a organização eficaz do estoque em depósitos. Projetado cuidadosamente para otimizar o gerenciamento de produtos, contagem física, auditoria e geração de relatórios, este aplicativo abrange diversas necessidades cruciais dos gerentes, auditores e equipes de controle de inventário. O aplicativo para dispositivo móvel proporciona funcionalidades específicas para auditores fazerem as contagem de produtos no depósito.

Para maximizar a utilidade do aplicativo, é crucial considerar a integração fluida com sistemas existentes na empresa, garantindo uma operação conjunta e eficiente. Além disso, características como rastreamento de movimentação de produtos, alerta automatizados, segurança robusta, histórico detalhado de alterações e suporte a tecnologias inovadoras, como leitura de códigos de barras, são incorporadas para aprimorar ainda mais a eficácia do sistema.

% Objetivos específicos

Os objetivos deste projeto incluem a implementação de um aplicativo de gestão de inventário, com ênfase na contagem e auditoria de estoques em depósitos, com o propósito de facilitar o controle e a organização dos produtos. O aplicativo deve integrar-se de maneira eficiente com os sistemas de gestão já utilizados pela empresa, assegurando a consistência e a atualização em tempo real dos dados de estoque.
O escopo do aplicativo é claramente definido, sendo desenvolvido para atender às regras de negócio específicas de uma empresa em particular. Não é o objetivo do projeto criar um aplicativo que seja compatível com os diversos tipos de inventário existentes; o desenvolvimento será focado exclusivamente nas necessidades e processos da empresa em questão.
Serão incorporadas funcionalidades como leitura de códigos de barras, rastreamento da movimentação de produtos e alerta automatizados, visando aprimorar a eficiência do processo de inventário. Adicionalmente, é imperativo criar uma interface amigável e intuitiva para auditores e gerentes, proporcionando uma experiência de usuário eficiente e adaptada às exigências da empresa.
Finalmente, é essencial garantir que os usuários recebam treinamento adequado e suporte técnico contínuo para assegurar uma adoção eficaz do sistema.
