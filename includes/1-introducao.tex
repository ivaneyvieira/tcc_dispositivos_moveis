% apresentar o problema que o aplicativo visa solucionar

No cenário dinâmico do varejo brasileiro, os erros de estoque representam uma questão crucial que pode impactar negativamente a eficiência operacional e a satisfação do cliente. O desafio reside na necessidade de manter um equilíbrio delicado entre a oferta e a demanda, garantindo que os produtos certos estejam disponíveis no momento certo. Nesse contexto, a implementação de um processo de inventário assume um papel estratégico e indispensável.

% Um breve referencial teórico

Realizar inventários é crucial para garantir a precisão das informações de saldo de estoque. Erros no registro de transações e no manuseio físico do estoque podem causar discrepâncias entre o estoque registrado e o real, que só são corrigidas durante verificações físicas esporádicas. Na prática, diversas transações aumentam a possibilidade de erros, resultando em registros imprecisos de estoque. As causas comuns incluem erros de digitação, contagem incorreta de produtos, falhas em registrar corretamente produtos danificados ou destruídos, retirada e retorno de itens sem a devida correção nos registros, atrasos na atualização dos registros após transações e roubos de estoque, frequentes no varejo e também presentes em ambientes industriais e comerciais \cite{nigel}.

Segundo a pesquisa realizada por \cite{silva}, a adoção de inventários cíclicos representa uma prática estratégica que não apenas melhora a acurácia e a gestão dos estoques, mas também promove a eficiência operacional e a redução de custos, fortalecendo a competitividade da empresa no mercado.

% Objetivo final da introdução

Um aplicativo de gestão de inventário emerge como uma solução indispensável para o controle e a organização eficaz do estoque em depósitos. Projetado cuidadosamente para otimizar o gerenciamento de produtos, contagem física, auditoria e geração de relatórios, este aplicativo abrange diversas necessidades cruciais dos gerentes, auditores e equipes de controle de inventário. Estruturado em dois módulos distintos, o aplicativo para dispositivo móvel que proporciona funcionalidades específicas para auditores fazerem as contagem de produtos no depósito, enquanto outro atende às demandas do gerente de inventário.

O módulo destinado aos auditores é implementado em dispositivos móveis, como coletores de dados, facilitando a execução da contagem física dos produtos nos diversos lotes do depósito. Por meio de uma interface intuitiva, os auditores podem realizar suas atividades de forma eficiente, assegurando a precisão das informações coletadas.

No âmbito do sistema de gerenciamento, acessível mediante um computador, o gerente do inventário encontra ferramentas abrangentes para cada etapa do ciclo de vida do produto. Desde o cadastro detalhado de produtos até o gerenciamento eficaz de lotes, auditoria e a geração de relatórios customizados, todas as operações são simplificadas para promover uma administração ágil e informada.

Para maximizar a utilidade do aplicativo, é crucial considerar a integração fluida com sistemas existentes na empresa, garantindo uma operação conjunta e eficiente. Além disso, características como rastreamento de movimentação de produtos, alerta automatizados, segurança robusta, histórico detalhado de alterações e suporte a tecnologias inovadoras, como RFID ou códigos de barras, são incorporadas para aprimorar ainda mais a eficácia do sistema.

Relatórios personalizáveis oferecem uma visão detalhada e adaptada às necessidades específicas da empresa, permitindo que as informações mais relevantes sejam destacadas e analisadas de maneira eficiente. Além disso, atualizações em tempo real, e acessibilidade móvel garantem que os dados estejam sempre atualizados e possam ser acessados de qualquer lugar, a qualquer momento, proporcionando uma tomada de decisão mais ágil e informada.

Em adição, a implementação de treinamentos adequados para os usuários, aliada à disponibilidade de suporte técnico, contribui para a adoção efetiva do aplicativo, maximizando seus benefícios operacionais. Com essas características, o aplicativo de gestão de inventário se destaca como uma ferramenta indispensável para aprimorar a eficiência e a transparência no gerenciamento de estoque.
