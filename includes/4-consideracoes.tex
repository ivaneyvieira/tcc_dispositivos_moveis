
Os erros de saldo de estoque no varejo brasileiro representam um desafio considerável que impacta tanto a eficiência operacional quanto a satisfação do cliente. Este estudo propõe a implementação de um processo de inventário robusto e contínuo, integrado à gestão operacional, como uma solução estratégica para este problema.

A adoção de tecnologias avançadas, tais como sistemas automatizados de rastreamento e leitura por código de barras, pode significativamente aumentar a eficácia do processo de inventário, reduzindo o tempo necessário para corrigir discrepâncias. Ferramentas como Flutter e Spring Boot evidenciam a viabilidade de desenvolver sistemas eficientes para a gestão de inventários.

Contudo, a tecnologia por si só não é suficiente para resolver todos os problemas relacionados ao saldo de estoque. É imperativo serem estabelecidos processos bem definidos, oferecido treinamento adequado e cultivada uma cultura organizacional que valorize a precisão. Dessa forma, ao adotar essas práticas, os varejistas não apenas reduzirão os erros de saldo de estoque, mas também aprimorarão a gestão do estoque e a experiência de compra dos clientes.