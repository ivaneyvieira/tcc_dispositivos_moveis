Este trabalho visa desenvolver um aplicativo de gestão de inventário para otimizar o controle e a organização de estoque em depósitos. No cenário dinâmico do varejo brasileiro, os erros de estoque são um desafio significativo, impactando a eficiência dos processos e a satisfação do cliente. A implementação de inventários cíclicos melhora a precisão e gestão dos estoques, promovendo a eficiência operacional e a redução de custos. O aplicativo proposto possui dois módulos: um para auditores, que permite a contagem de produtos via dispositivos móveis, e outro para gerentes, oferecendo ferramentas abrangentes para cadastro, gestão de lotes, auditoria e geração de relatórios. Com integração a sistemas existentes, rastreamento de movimentação de produtos e suporte a leitura de códigos de barras, o aplicativo visa maximizar a eficiência e transparência no gerenciamento de estoque. Treinamentos e suporte técnico contínuos garantem a adoção eficaz do sistema, destacando-se como uma solução essencial para empresas de diversos setores.

\textbf{Palavras-chave}: Gestão de Inventário. Eficiência Operacional. Auditoria de Estoque. Aplicativo Móvel.