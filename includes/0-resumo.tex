Este trabalho apresenta o desenvolvimento de um aplicativo móvel utilizando coletores de dados, focado na contagem de produtos para a correção de saldo de estoque em depósitos. A metodologia Kanban foi empregada para o levantamento de requisitos, permitindo um desenvolvimento ágil e flexível. O aplicativo foi desenvolvido utilizando Flutter para a camada de apresentação e Spring Boot para a camada de serviços, com banco de dados MySQL. As principais funcionalidades incluem a integração com sistemas de gestão existentes, rastreamento de movimentação de produtos, leitura de códigos de barras e alerta automatizados. A interface foi projetada para ser intuitiva, facilitando o uso por gerentes e auditores. Durante o desenvolvimento, foi dado enfoque à experiência do usuário, utilizando o Figma para prototipação e testes de usabilidade. Os resultados indicam que a adoção de tecnologias modernas pode aumentar significativamente a eficiência do processo de inventário, reduzindo erros e melhorando a gestão de estoques. Conclui-se que, além da tecnologia, é essencial estabelecer processos bem definidos e fornecer treinamento adequado aos usuários para garantir a eficácia do sistema. O estudo destaca a importância de uma abordagem integrada, combinando tecnologia com práticas de gestão bem estruturadas para otimizar o controle de inventário em depósitos.

\textbf{Palavras-chave}: Gestão de Inventário. Aplicativos Móveis. Coletores de Dados. Metodologia Kanban.