%Este trabalho visa desenvolver um aplicativo de gestão de inventário para otimizar o controle e a organização de estoque em depósitos. No cenário dinâmico do varejo brasileiro, os erros de estoque são um desafio significativo, impactando a eficiência dos processos e a satisfação do cliente. A implementação de inventários cíclicos melhora a precisão e gestão dos estoques, promovendo a eficiência operacional e a redução de custos. O aplicativo proposto será responsável pela contagem de inventário com integração a sistemas existentes, usando boas práticas de usabilidade, em um coletor de dados Android, suporte a leitura de códigos de barras e objetivando maximizar a eficiência e transparência no gerenciamento de estoque.

O presente trabalho aborda o desenvolvimento de um aplicativo de gestão de inventário, focado na otimização do controle de estoques em depósitos. A precisão no registro e na contagem de produtos é fundamental para minimizar erros de estoque que impactam negativamente a eficiência operacional e a satisfação dos clientes. Para isso, foi desenvolvido um aplicativo móvel utilizando Flutter para o frontend e Spring Boot para o backend, integrando-se ao sistema de gestão existente da empresa. O aplicativo possibilita aos auditores realizar contagens físicas de produtos, registrar dados em tempo real e gerar relatórios detalhados, facilitando a tomada de decisões gerenciais. A implementação de tecnologias como leitura de códigos de barras e alerta automáticos são algumas das funcionalidades que aumentam a eficácia do sistema. A adoção deste aplicativo visa reduzir custos operacionais, melhorar a precisão dos dados de estoque e aumentar a eficiência na gestão de inventários.

\textbf{Palavras-chave}: Gestão de Inventário. Eficiência Operacional. Auditoria de Estoque. Aplicativo Móvel.