% apresentar o problema que o aplicativo visa solucionar

No cenário dinâmico do varejo brasileiro, os erros de estoque representam uma questão crucial que pode impactar negativamente a eficiência operacional e a satisfação do cliente. O desafio reside na necessidade de manter um equilíbrio delicado entre a oferta e a demanda, garantindo que os produtos certos estejam disponíveis no momento certo. Nesse contexto, a implementação de um processo de inventário assume um papel estratégico e indispensável.

% Um breve referencial teórico

Realizar inventários é crucial para garantir a precisão das informações de saldo de estoque. Erros no registro de transações e no manuseio físico do estoque podem causar discrepâncias entre o estoque registrado e o real, que só são corrigidas durante verificações físicas esporádicas. Na prática, diversas transações aumentam a possibilidade de erros, resultando em registros imprecisos de estoque. As causas comuns incluem erros de digitação, contagem incorreta de produtos, falhas em registrar corretamente produtos danificados ou destruídos, retirada e retorno de itens sem a devida correção nos registros, atrasos na atualização dos registros após transações e roubos de estoque, que são frequentes no varejo e também presentes em ambientes industriais e comerciais \cite{nigel}.

Segundo a pesquisa realizada por \cite{silva}, a adoção de inventários cíclicos representa uma prática estratégica que não apenas melhora a acurácia e a gestão dos estoques, mas também promove a eficiência operacional e a redução de custos, fortalecendo a competitividade da empresa no mercado.

% Objetivo final da introdução
