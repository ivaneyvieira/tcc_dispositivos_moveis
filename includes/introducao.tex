% apresentar o problema que o aplicativo visa solucionar

Para manter a competitividade na era da globalização, as empresas de varejo adotam novas 
tecnologias e processos organizacionais, focando no controle de inventário e na armazenagem 
eficiente. Gerenciar recursos materiais, saber onde e quanto estocar, e escolher os meios 
de transporte adequados são essenciais para agregar valor ao negócio. Técnicas inovadoras 
e ferramentas de gestão de estoque são exploradas para otimizar esses processos. Um bom 
controle de inventário é crucial para garantir a eficiência operacional e a satisfação do
cliente.
% Um breve referencial teórico

Os inventários servem como meio de ajudar o controle dos estoques, tornando
confiáveis as informações, facilitando que as operações aconteçam de forma ágil e com
eficiência. Desta forma, conduzir o estoque é ter controle por meio de recursos
disponíveis, que forneçam fatos para tomada de decisões e acompanhamento de toda
movimentação desde a compra até e venda, com o propósito em evitar prejuízos para
empresa (Slack 2009).

Segundo a pesquisa realizada por Eliane Francisca Silva em 2022, foi possível 
identificar a importância da implementação de inventários cíclicos para garantir 
uma melhor acurácia e gestão dos estoques, já que sua aplicação resulta na redução 
dos custos operacionais da empresa.

% Objetivo final da introdução