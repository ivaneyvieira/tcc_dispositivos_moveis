\subsection{Funcionalidades do aplicativo}

\begin{figure}[H]
    \centering
    \subfigure[Tela de login. \label{fig:login}]{
        \includegraphics[width=0.15\textwidth]{imgs/login.png}
    }
    \quad
    \subfigure[Tela principal. \label{fig:homepage}]{
        \includegraphics[width=0.15\textwidth]{imgs/homepage.png}
    }
    \quad
    \subfigure[Seleção de inventário.\label{fig:inventario}]{
        \includegraphics[width=0.15\textwidth]{imgs/inventario.png}
    }
    \quad
    \subfigure[Seleção de Lote.\label{fig:lote}]{
        \includegraphics[width=0.15\textwidth]{imgs/lote.png}
    }
    \quad
    \subfigure[Coleta para contagem.\label{fig:coleta}]{
        \includegraphics[width=0.15\textwidth]{imgs/coleta.png}
    }
    \label{fig:grupo01}
    \caption{Layout de telas para o aplicativo}
\end{figure}

Na figura \ref{fig:login} temos a tela de login onde o usuário deve informar o usuário e senha para acessar o aplicativo. Lembrando que o usuário é autenticado usando as mesmas informações de acesso do ERP da empresa. A figura \ref{fig:homepage} mostra a tela principal do aplicativo, onde o usuário pode selecionar o inventário que deseja realizar a contagem. A figura \ref{fig:inventario} mostra a tela de seleção de inventário, onde o usuário pode visualizar os inventários disponíveis e selecionar o que deseja realizar a contagem. A figura \ref{fig:lote} mostra a tela de seleção de lote, onde o usuário pode visualizar os lotes disponíveis e selecionar o que deseja realizar a contagem. A figura \ref{fig:coleta} mostra a tela de coleta para contagem, onde o usuário pode visualizar os produtos disponíveis no lote e informar a quantidade contada de cada produto.

\subsection{Aplicativos semelhantes}

\begin{table}[H]
    \centering
    \caption{Funcionalidades dos aplicativos semelhantes.}
    \label{tab:comparativos}
    \includegraphics[width=1.0\textwidth]{tables/comparativo.png}
\end{table}

O IS Collector se destaca pela flexibilidade e abrangência de funções, ideal para empresas que exigem um alto nível de controle e automação no processo de inventário. O KCollector é a opção ideal para quem busca uma solução acessível e prática, que transforma o celular em um coletor eficiente. Já o Stock e Inventário Simples se destaca pela facilidade de uso e pelo gerenciamento completo do estoque, sendo uma ótima opção para iniciantes e pequenos negócios.

% Colocar aqui um parágrafo comparando o aplicativo do trabalho com os aplicativos pesquisados
O aplicativo do presente trabalho se destaca pela integração com o sistema de gestão da empresa, garantindo a consistência e a atualização em tempo real dos dados de estoque. Além disso, a interface intuitiva e as funcionalidades customizáveis proporcionam uma experiência de usuário eficiente e adaptável às necessidades específicas de cada empresa. A integração com tecnologias avançadas, como leitura de códigos de barras e alertas automatizados, eleva a eficácia do aplicativo, tornando-o uma solução completa e moderna para o controle de inventário.





