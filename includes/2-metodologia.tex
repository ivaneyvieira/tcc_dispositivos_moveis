% Descrever detalhadamente a metodologia utilizada no desenvolvimento do aplicativo,

\subsection{Abordagem de desenvolvimento}

% * Abordagem de desenvolvimento: (Ex: Ágil, Waterfall)

%Os métodos ágeis surgiram para corrigir deficiências percebidas e reais da engenharia de software tradicional.
Os métodos ágeis corrigem deficiências da engenharia de software tradicional.
Embora ofereçam benefícios significativos, não são universais e não contradizem completamente as práticas confiáveis de engenharia de software. Eles podem ser aplicados como uma abordagem geral para todos os tipos de projetos de software \cite{pressman}.
%Os métodos ágeis surgiram para corrigir deficiências percebidas e reais da engenharia de software tradicional. Embora oferecessem benefícios significativos, não eram universais e não contradiziam completamente as práticas confiáveis de engenharia de software. Eles podiam ser aplicados como uma abordagem geral para todos os tipos de projetos de software \cite{pressman}.

% Alerta de plagio
Na economia atual, é frequentemente difícil prever o desenvolvimento de sistemas computacionais, como aplicativos móveis. Mudanças rápidas ocorrem no mercado, as demandas dos consumidores são alteradas, e novas ameaças competitivas surgem sem aviso. Muitas vezes, é impraticável estabelecer completamente os requisitos antes do início do projeto. É crucial ter agilidade o bastante para se adaptar a um ambiente de negócios dinâmico \cite{pressman}.
%Na economia moderna, foi frequentemente difícil ou impossível prever a evolução de sistemas computacionais, como aplicativos móveis. As condições de mercado mudavam rapidamente, as necessidades dos usuários se transformavam, e novas ameaças competitivas surgiam sem aviso. Em muitos casos, foi impossível definir completamente os requisitos antes do início do projeto. Portanto, foi essencial ser ágil o suficiente para se adaptar a um ambiente de negócios dinâmico \cite{pressman}.

O Kanban é uma metodologia ágil amplamente adotada devido à sua capacidade de visualizar e gerenciar eficientemente fluxos de trabalho. Ele proporciona transparência sobre o progresso das tarefas e limita o trabalho em progresso, permitindo que a equipe se concentre em concluir tarefas antes de iniciar novas. Com colunas como “Para Fazer”, “Em Progresso” e “Concluído”, o Kanban facilita a priorização contínua baseada nas necessidades atuais do projeto e permite ajustes rápidos conforme novos requisitos emergem ou mudam. Essa flexibilidade é crucial em um ambiente onde as demandas do mercado e dos usuários podem evoluir rapidamente, garantindo que o desenvolvimento seja adaptável e responsivo às necessidades reais.
%O Kanban foi uma metodologia ágil amplamente adotada devido à sua capacidade de visualizar e gerenciar eficientemente fluxos de trabalho. Ele proporcionou transparência sobre o progresso das tarefas e limitou o trabalho em progresso (WIP), permitindo que a equipe se concentrasse em concluir tarefas antes de iniciar novas. Com colunas como “Para Fazer”, “Em Progresso” e “Concluído”, o Kanban facilitou a priorização contínua baseada nas necessidades atuais do projeto e permitiu ajustes rápidos conforme novos requisitos emergiam ou mudavam. Essa flexibilidade foi crucial em um ambiente onde as demandas do mercado e dos usuários podiam evoluir rapidamente, garantindo que o desenvolvimento fosse adaptável e responsivo às necessidades reais.

Essa integração do Kanban com os princípios ágeis fortalece a capacidade do desenvolvimento de software de responder de maneira ágil e eficaz às mudanças, mantendo ao mesmo tempo, um controle rigoroso sobre o progresso e a qualidade do produto final.
%Essa integração do Kanban com os princípios ágeis fortaleceu a capacidade do desenvolvimento de software de responder de maneira ágil e eficaz às mudanças, mantendo ao mesmo tempo um controle rigoroso sobre o progresso e a qualidade do produto final.

\subsection{Ferramentas e tecnologias}

O levantamento de requisitos foi feito por meio de entrevista com os usuários chaves do setor de estoque e inventário da empresa, onde foram identificadas as necessidades e funcionalidades essenciais para o aplicativo. A partir dessas informações, foi elaborado um documento de especificação de requisitos, que serviu como base para o desenvolvimento do aplicativo.

Além das entrevistas, foi realizada uma análise comparativa detalhada dos principais aplicativos de contagem de inventários disponíveis na loja de aplicativos do Google: IS Collector, KCollector e o Stock e Inventário Simples. A análise abrangeu funcionalidades, vantagens e desvantagens de cada aplicativo.

%\subsubsection{Frontend}

%No frontend, será utilizado o Figma para criar protótipos de design e interfaces de usuário, permitindo uma visualização clara e interativa do aplicativo antes do desenvolvimento real. O Figma é uma ferramenta de design colaborativa baseada na web, usada para criar interfaces de usuário, protótipos e gráficos vetoriais. Ele facilita a comunicação e a iteração entre designers e desenvolvedores em tempo real, garantindo que todos estejam alinhados durante o processo de design.
No frontend, foi utilizado o Figma (\url{https://www.figma.com}) para criar protótipos de design e interfaces de usuário, permitindo uma visualização clara e interativa do aplicativo antes do desenvolvimento real. O Figma é uma ferramenta de design colaborativa baseada na web, usada para criar interfaces de usuário, protótipos e gráficos vetoriais. Ele facilita a comunicação e a iteração entre designers e desenvolvedores em tempo real, garantindo que todos estivessem alinhados durante o processo de design.

%Para o desenvolvimento da interface do aplicativo, será usado o Flutter, um kit de desenvolvimento de software (SDK) criado pelo Google. O Flutter permite a construção de aplicativos nativos de alta performance para iOS, Android, web e desktop a partir de uma única base de código. Utilizando a linguagem Dart, o Flutter é conhecido por sua capacidade de criar interfaces de usuário bonitas e interativas rapidamente, proporcionando uma experiência de usuário consistente e responsiva em múltiplas plataformas.

Para o desenvolvimento da interface do aplicativo, foi usado o Flutter versão 3.22 (\url{https://flutter.dev}), um kit de desenvolvimento de software criado pelo Google. O Flutter permitiu a construção de aplicativos nativos de alto desempenho para iOS, Android, web e desktop a partir de uma única base de código. Utilizando a linguagem Dart versão 3.4 (\url{https://dart.dev}), o Flutter é conhecido por sua capacidade de criar interfaces de usuário bonitas e interativas rapidamente, proporcionando uma experiência de usuário consistente e responsiva em múltiplas plataformas.

%O Android Studio e o emulador de Android serão utilizados para desenvolver, testar e depurar o aplicativo em um ambiente controlado que simula dispositivos Android reais. O Android Studio oferece um conjunto completo de ferramentas para o desenvolvimento Flutter, incluindo um editor de código, ferramentas de depuração e um emulador integrado.
O Android Studio versão 2024.1.1 (\url{https://developer.android.com/studio}) foi utilizado para desenvolver, testar e depurar o aplicativo. O Android Studio oferece um conjunto completo de ferramentas para o desenvolvimento Flutter, incluindo um editor de código, ferramentas de depuração e um emulador integrado.

%No backend, será utilizado o Spring Boot para desenvolver a estrutura do aplicativo. O Spring Boot é um framework Java que facilita a criação de aplicativos stand-alone e production-ready, simplificando a configuração e a implementação de serviços backend. Ele oferece uma estrutura robusta e escalável para criar e gerenciar APIs RESTful, serviços web e lógica de negócios.
%
%O MySQL será o sistema de gerenciamento de banco de dados escolhido para armazenar e recuperar dados de forma eficiente e segura. O MySQL é um sistema de gerenciamento de banco de dados relacional (RDBMS) de código aberto, usado para armazenar, organizar e acessar dados de maneira eficiente, permitindo operações complexas de consulta e manipulação de dados.
%
%Além disso, o IntelliJ IDEA será utilizado como ambiente de desenvolvimento integrado (IDE) para escrever, depurar e testar o código backend. O IntelliJ IDEA é uma das IDEs mais populares e poderosas para o desenvolvimento Java, oferecendo uma ampla gama de ferramentas e funcionalidades que aumentam a produtividade do desenvolvedor.

%\subsubsection{Backend}

%No backend, será utilizado o Spring Boot para desenvolver a estrutura do aplicativo. O MySQL será o sistema de gerenciamento de banco de dados escolhido para armazenar e recuperar dados de forma eficiente e segura. Além disso, o IntelliJ IDEA será utilizado como ambiente de desenvolvimento integrado (IDE) para escrever, depurar e testar o código backend. No entanto, a implementação detalhada do backend não é o foco principal deste trabalho.

O backend inclui a API Rest Full, que faz a integração do Aplicativo com o ERP da organização. Ele foi implementado na linguagem Java versão 21 (\url{https://www.oracle.com/java/}) com o framework Spring Boot versão 3.2 (\url{https://spring.io/projects/spring-boot}).
O IntelliJ IDEA versão 2024.1.4 (\url{https://www.jetbrains.com/idea/}) foi utilizado como ambiente de desenvolvimento integrado (IDE) para escrever, depurar e testar o código backend. 
Além disso, O MySQL versão 8.3 (\url{https://www.mysql.com}) foi o sistema de gerenciamento de banco de dados escolhido para armazenar e recuperar dados de forma eficiente e segura. 
No entanto, a implementação detalhada do backend não é o foco principal deste trabalho.

%\subsection{Coleta de dados}
%
%Coleta de dados (caso tenha sido utilizada): (Ex: Entrevistas, questionários, testes de usabilidade)
%
%\subsection{Análise de dados}
%
%Análise de dados (se for utilizada no trabalho): (Ex: Métodos estatísticos, análise de conteúdo)
