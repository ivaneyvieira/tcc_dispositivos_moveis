% Descrever detalhadamente a metodologia utilizada no desenvolvimento do aplicativo,

\subsection{Abordagem de desenvolvimento}

Os métodos ágeis corrigem deficiências da engenharia de software tradicional.
Embora ofereçam benefícios significativos, não são universais e não contradizem completamente as práticas confiáveis de engenharia de software. Eles podem ser aplicados como uma abordagem geral para todos os tipos de projetos de software \cite{pressman}.

Na economia atual, é frequentemente difícil prever o desenvolvimento de sistemas computacionais, como aplicativos móveis. Mudanças rápidas ocorrem no mercado, as demandas dos consumidores são alteradas, e novas ameaças competitivas surgem sem aviso. Muitas vezes, é impraticável estabelecer completamente os requisitos antes do início do projeto. É crucial ter agilidade o bastante para se adaptar a um ambiente de negócios dinâmico \cite{pressman}.

A metodologia escolhida foi o Kanban, por ser uma metodologia ágil amplamente adotada devido à sua capacidade de visualizar e gerenciar eficientemente fluxos de trabalho. Ele proporciona transparência sobre o progresso das tarefas e limita o trabalho em progresso, permitindo que a equipe se concentre em concluir tarefas antes de iniciar novas.

Na aplicação da metodologia Kanban para o levantamento de requisitos, é imperativo adotar práticas ágeis e integrar processos de iteração e feedback para assegurar um desenvolvimento flexível e eficiente. Inicialmente, compreender as necessidades dos envolvidos é fundamental. Esse entendimento foi obtido por meio de entrevistas e questionários, os quais forneceram uma visão clara das expectativas e exigências do sistema. Com base nessa compreensão, foi elaborado um backlog de requisitos, no qual cada item representa uma funcionalidade ou necessidade específica. Este backlog é então priorizado conforme o entendimento obtido em reuniões com a equipe de desenvolvimento e as partes interessadas no projeto.

Na aplicação da metodologia Kanban para levantamento de requisitos, é crucial adotar práticas ágeis e integrar processos de iteração e feedback para garantir um desenvolvimento flexível e eficiente. O entendimento das necessidades dos envolvidos foi obtido por meio de entrevistas e questionários, resultando na criação e priorização de um backlog de requisitos. Embora o Kanban não utilize ciclo de desenvolvimento fixos, foram estabelecidas reuniões semanais com as equipes de desenvolvimento, permitindo que cada item seja movido através das fases do quadro Kanban: “A Fazer”, “Em Andamento” e “Concluído”.

Reuniões regulares de revisão e retrospectiva foram realizadas para avaliar o progresso e obter feedback das partes interessadas. Esse feedback contínuo permite ajustar os requisitos e o desenvolvimento conforme necessário. O backlog de requisitos é refinado com base nas observações das iterações, incluindo a adição de novos requisitos e a remoção de itens desatualizados. Documentar mudanças e manter uma comunicação clara entre a equipe e as partes interessadas é essencial para alinhar os requisitos e o progresso do projeto.

Em resumo, a combinação da metodologia Kanban com práticas ágeis e processos de iteração e feedback permite um levantamento de requisitos adaptável e eficiente, garantindo que o desenvolvimento responda rapidamente às mudanças, enquanto mantém um controle rigoroso sobre o progresso e a qualidade do produto final.

\subsection{Ferramentas e tecnologias}

%\subsubsection{Frontend}

No desenvolvimento do frontend, foi utilizado o Figma (\url{https://www.figma.com}) para a criação de protótipos de design e interfaces de usuário, permitindo uma visualização clara e interativa do aplicativo antes do desenvolvimento propriamente dito. O Figma é uma ferramenta de design colaborativa baseada na web, empregada para a criação de interfaces de usuário, protótipos e gráficos vetoriais. A ferramenta facilita a comunicação e a iteração entre designers e desenvolvedores em tempo real, assegurando que todos estejam alinhados durante todo o processo de design.

Para a criação da interface do aplicativo, foi empregado o Flutter na versão 3.22 (\url{https://flutter.dev}), um kit de desenvolvimento de software criado pelo Google. O Flutter possibilitou a construção de aplicativos nativos de alto desempenho para iOS, Android, web e desktop a partir de uma única base de código. Utilizando a linguagem Dart na versão 3.4 (\url{https://dart.dev}), o Flutter é renomado por sua capacidade de criar interfaces de usuário belas e interativas de forma ágil, proporcionando uma experiência de usuário consistente e responsiva em múltiplas plataformas.

O Android Studio versão 2024.1.1 (\url{https://developer.android.com/studio}) foi utilizado para desenvolver, testar e depurar o aplicativo. Esta ferramenta oferece um conjunto completo de recursos para o desenvolvimento com Flutter, incluindo um editor de código, ferramentas de depuração e um emulador integrado.

%\subsubsection{Backend}

O backend inclui uma API RESTful, que realiza a integração do aplicativo com o sistema de gestão da organização. Foi desenvolvido na linguagem Java, versão 21 (\url{https://www.oracle.com/java/}), utilizando o framework Spring Boot, versão 3.2 (\url{https://spring.io/projects/spring-boot}).
O IntelliJ IDEA, versão 2024.1.4 (\url{https://www.jetbrains.com/idea/}), foi empregado como ambiente de desenvolvimento integrado para a escrita, depuração e teste do código backend. Adicionalmente, o MySQL, versão 8.3 (\url{https://www.mysql.com}), foi selecionado como sistema de gerenciamento de banco de dados, garantindo armazenamento e recuperação de dados de forma eficiente e segura.
