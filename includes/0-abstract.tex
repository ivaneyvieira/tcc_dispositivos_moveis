%This work aims to develop an inventory management application to optimize the control and organization of stock in warehouses. In Brazil’s dynamic retail landscape, inventory errors are a significant challenge, impacting process efficiency and customer satisfaction. Implementing cyclical inventories improves stock accuracy and management, promoting operational efficiency and reducing costs. The proposed application will be responsible for inventory counting with integration into existing systems, using good usability practices, on an Android data collector, supporting barcode scanning and aiming to maximize efficiency and transparency in inventory management.

This paper deals with the development of an inventory management application focused on optimizing stock control in warehouses. Accuracy in recording and counting products is essential for minimizing inventory errors that negatively impact operational efficiency and customer satisfaction. To this end, a mobile application was developed using Flutter for the frontend and Spring Boot for the backend, integrating with the company’s existing management system. The app enables auditors to carry out physical product counts, record data in real time and generate detailed reports, facilitating management decision-making. The implementation of technologies such as barcode scanning and automatic alerts are some of the features that increase the system’s effectiveness. Adopting this application aims to reduce operating costs, improve the accuracy of stock data and increase efficiency in inventory management.

%\textbf{Keywords}: Inventory Management. Operational efficiency. Stock Audit. Mobile Application.